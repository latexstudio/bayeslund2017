\begin{frame}
	\frametitle{Some lessons learned}

	\begin{itemize}
		\item Delving into mathematical details, e.g. derivations of formulae for posterior distributions, did not prove to be very effective.
		\item Learning by building and running Jags models proved much more effective.
		\item Being comfortable with R is vital. Pre-workshop R bootcamps were popular and effective.
		\item Software installation problems can stymie progress.
		\item For many attendees, Bayesian data analysis means Bayesian hypothesis testing (with Bayes factors). While for us, Bayesian data analysis is more about flexible probabilistic modelling.
		\item The age of Bugs/Jags has (probably) passed, Stan is now the preferred choice as a probabilistic modelling language.

	\end{itemize}

\end{frame}

\begin{frame}
	\frametitle{Recommendations for the future syllabus}


	\begin{itemize}
		\item There seem to be three levels of Bayesian data analysis that people want to learn:

	\begin{itemize}

		\item \emph{Introductory}: The fundamentals of Bayesian data analysis and Bayesian hypothesis testing.
		\item \emph{Intermediate}: Regression modelling, i.e. linear, general linear, generalized linear, multilevel regression, including robust regression, model checking, model evaluation.
		\item \emph{Advanced}:  A wide set of topics including: Nonlinear regression, latent variable modelling, mixture modelling, time series modelling, and possibly also causal modelling, structural equation modelling, Bayesian machine learning (Bayesian deep learning), etc.
	\end{itemize}

	\end{itemize}

\end{frame}


